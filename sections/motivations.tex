\documentclass[../main.tex]{subfiles}
\begin{document}
Computer Science is arguably one of the most interesting fields of \textbf{math}, so why is there such a big disconnect between the math one learns in university and high school CS and the programming language of choice?

Take the example of a simple sum function in math:

\[ f(a, b) = \sum^a_{b} b , \set{a,b} \in \mathbb{N}  \]

This is an example of Sigma notation, a much-used concept in many parts of math. Taking a look at the equivalent expression in Python,

\begin{lstlisting}[language=python]
    def sum_loop (top_bound, start):
        accumulator = 0
        for i in range(start, top_bound+1):
            accumulator += i
            i+=1
        return accumulator
\end{lstlisting}

One might see that the two have absolutely nothing in common.

This may not seem like a problem; the programmer just needs to learn programmatical intuition. However, this can pose a challenge for a \textit{mathematician} to learn programming, because of the existing mathematical intuition that needs to be replaced
\end{document}