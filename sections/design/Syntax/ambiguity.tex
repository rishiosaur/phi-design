\documentclass[../../main.tex]{subfiles}
\begin{document}
$\Phi$ is a language that doesn't allow for any ambiguity. All characters and keywords must have \textit{one} purpose.

For instance, in Swift, the \texttt{:} operator serves at least functions: One for type annotation, and the other for assigning values:

\begin{verbatim}
//Use one: Type annotation
var x : String = "Hello World"

func printString(str : String) -> Void {
    print(str)
}

//Use two: Value assignment
printString(str: x)
\end{verbatim}

The fact that the \texttt{:} operator can be used in multiple cases creates ambiguity, and $\Phi$ aims to solve that.

In $\Phi$'s case, the \texttt{:} operator will only be used for type annotation, and the \texttt{=} operator will be used as an assignment operator.

Similarly, the \texttt{?} operator will serve as an equality operator, in comparison statements:

\begin{verbatim}
    if(equality: { x ?= y }, action: {
        print("x equals y")
    })

    if(equality: { x ?: y })
\end{verbatim}

\end{document}